\section{Задание 10: Контроль за реализацией проекта}

Установим дату отчена на 01.06.2023.

\imgw{task10-01}{h!}{7cm}{Дата отчета}

Учтем следующую фактическую информацию:

\begin{itemize}
    \item разработка модулей обработки данных завершилась
        24.05.2023~(рисунок~\ref{img:task10-02});

    \imgw{task10-02}{h!}{10cm}{Учет фактических параметров 1}

    \item подготовка тестовых наборов данных завершилась на 3 дня
        раньше~(рисунок~\ref{img:task10-03});

    \imgw{task10-03}{h!}{10cm}{Учет фактических параметров 2}

    \item тестирование и исправление ошибок выполнено на
        30\%~(рисунок~\ref{img:task10-04});

    \imgw{task10-04}{h!}{10cm}{Учет фактических параметров 3}

    \item во время перерыва между задачами в мае ведущий программист прошел
        курсы повышения квалификации и его зарплата увеличилась на
        10\%~(рисунок~\ref{img:task10-05}).

    \imgw{task10-05}{h!}{13cm}{Учет фактических параметров 4}
\end{itemize}

Все остальные задачи установим выполненными на столько, на сколько они должны
быть выполнены на дату отчета~(рисунок~\ref{img:task10-06}).

\imgw{task10-06}{h!}{12cm}{Установка остальных задач по дате отчета}

Отклонения от базового плана и линия прогресса представлены на
рисунках~\ref{img:task10-08}-\ref{img:task10-07}. 

\imgw{task10-08}{h!}{12cm}{Статистика проекта с учетом фактических параметров}

\imgw{task10-07}{h!}{17cm}{Линия програсса}

Параметры проекта по методике освоенного объема приведены на
рисунке~\ref{img:task10-09}.

\imgw{task10-09}{h!}{17cm}{Параметры проекта по методике освоенного объема}

\subsection*{Вывод}

С учетом введенных фактических параметров срок проекта сдвинулся на 17.07.2023,
затраты увеличились на 2~500~рублей, однако оба параметра соответсвуют
установленным требованиям к проекту.

При этом:
\begin{itemize}
    \item $\text{ОКП} < 0$ --- проект выполняется с отставанием;
    \item $\text{ОПС} > 0$ --- проект укладывается в смету, за счет этих
        средств можно выделить дополнительные ресурсы, чтобы ликвидировать
        отставание;
    \item $\text{ОПЗ} > 0$ --- весь проект должен уложиться в смету, нет
        перерасхода средств.
\end{itemize}

