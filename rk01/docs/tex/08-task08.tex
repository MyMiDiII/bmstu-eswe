\section{Задание 8: Включение периодической задачи в план проект}

Задаем повторяющуюся задачу в соответствии с условием~(рисунок~\ref{img:task08-01}).

\imgw{task08-01}{h!}{12cm}{Сведения о повторяющейся задаче}

Добавляем специалиста по профилактике как материальный ресурс со станартной
ставкой 3000~рублей за один выезд~(рисунок~\ref{img:task08-02}) и назначаем его
на повторяющуюся задачу~(рисунок~\ref{img:task08-04}).

\imgw{task08-02}{h!}{12cm}{Добавление специалиста по профилактическим работам}

\imgw{task08-04}{h!}{12cm}{Назначение ресурса на повторяющуюся задачу}

Для того, чтобы никакие задачи во время выполнения периодической не
выполнялись, сократим рабочее время в эти дни на
3~часа~(рисунок~\ref{img:task08-03}).

\imgw{task08-03}{h!}{12cm}{Сокращение рабочих дней во время профилактики}

После проведенных изменений окончание проекта сдвинулось на 20.07.2023, затраты
не изменились~(рисунок~\ref{img:task08-05}).

\imgw{task08-05}{h!}{12cm}{Диаграмма Ганта после введения периодической задачи}

\subsection*{Вывод}

