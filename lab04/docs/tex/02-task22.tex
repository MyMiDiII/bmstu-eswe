\section{Влияние периодических задач}

Состояние проекта после добавления периодических задач без оптимизации
представлено на рисунке~\ref{img:lab03-task02-06}.

\imgw{lab03-task02-06}{h!}{17cm}{Длительность и бюджет проекта
после введения совещаний}

\vspace{1cm}

Введение периодической задачи <<Совещание>>, происходящей каждую среду
длительностью в один час, привело к значительному увеличению затрат
(на~20~039~рублей) так, что они составили~68~119~рублей~(на~18~119~рублей больше
заявленного бюджета). Также с этим произошла перегрузка ресурсов, после
выравнивания которых длительность проекта увеличилась на 5~дней и дата окончания
проекта стала 25.09.2023, что превышает установленные сроки на 25~дней.

\section{Оптимизация затрат}

Для уменьшения затрат было учтено, что во время совещаний сотрудники не занимают
свои рабочие места, поэтому были исключены из каждого используемого на совещании
трудового ресурса затраты на использование путем добавления новых таблиц норм
затрат к ресурсам.

Таким образом были получены затраты в~49~849~рублей, что укладывается в бюджет.
Результаты оптимизации затрат приведены на рисункe~\ref{img:lab03-task02-09}).

\imgw{lab03-task02-09}{h!}{17cm}{Результат оптимизации финансовых параметров}
