\section{Влияние периодических задач}

Введение периодической задачи <<Совещание>>, происходящей каждую среду
длительностью в один час, привело к значительному увеличению затрат
(на~20~039~рублей) так, что они составили~68~115~рублей~(на~18~115~рублей больше
заявленного бюджета). Также с этим произошла перегрузка ресурсов, после
выравнивания которых длительность проекта увеличилась на 7~дней и дата окончания
проекта стала 27.09.2023, что превышает установленные сроки на 27~дней.
Состояние проекта после добавления периодических задач без оптимизации
представлено на рисунке~\ref{img:lab04-not-optimized}.

%\imgw{lab04-not-optimized}{h!}{17cm}{Диграмма Ганта проекта после добавления
%<<Совещания>>}

%Для устранения перегрузки ресурсов проведем автоматическое выравнивание еще
%раз. После выравнивания в задачи были добавлены <<перерывы>> на совещание.
%Дата окончания проекта стала 27.09.2023~(рисунок~\ref{img:lab03-task02-06}).
%
%\imgw{lab03-task02-06}{h!}{17cm}{Результат выравнивания после введения совещаний}

%Во время совещаний сотрудники не занимают свои рабочие места, поэтому можно
%исключить из каждого используемого на совещании трудового ресурса затраты на
%использование и ввести фиксированные затраты на совещания, которые будут
%меньше.
%
%Исключение затрат на использование проводится путем добавление таблицы норм
%затрат~B к каждому ресурсу~(рисунок~\ref{img:lab03-task02-07}). Указание для
%каждого ресурса каждого совещания таблицы норм затрат~B производится через
%одноименный столбец представления \texttt{Использование
%задач}~(рисунок~\ref{img:lab03-task02-08}).
%
%\imgw{lab03-task02-07}{h!}{12cm}{Добавление новой таблицы норм затрат}
%
%\imgw{lab03-task02-08}{h!}{15cm}{Установка таблицы норм затрат ресурсу по
%конретной задаче}
%
%Итого получим затраты в размере~49~849~рублей, что укладывается в
%бюджет~(рисунок~\ref{img:lab03-task02-09}).
%
%\imgw{lab03-task02-09}{h!}{17cm}{Результат оптимизации финансовых параметров}
