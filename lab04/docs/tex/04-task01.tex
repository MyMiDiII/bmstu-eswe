\clearpage
\section{Актуализация параметров проекта}

\subsection{Текущие параметры проекта}

Дата отчета: 17.04.2023

На эту дату завершились все работы, которые должны были завершиться, кроме:

\begin{itemize}
    \item работа 6 завершилась на неделю позже;
    \item работа 9 завершилась на 10 дней позже;
    \item работа 17 выполнена на 10\%;
    \item с 1~апреля на 2~недели ведущий программист уехал на курсы повышения
	  квалификации, после возвращения с которых его заработная плата
	  поднята на~10\%;
    \item с 10~апреля совещания посещают только те сотрудники, которые
        выполняют работы на данный момент или будут выполнять работу через
        2~недели и меньше; после окончания своих работ сотрудники совещания не
        посещают.
\end{itemize}

\subsection{Настройка параметров проекта}

Задаем дату отчета во вкладке \texttt{Проект -> Состояние -> Дата отчета о
состоянии}, что представлено на рисунке~\ref{img:lab04-01}.

\imgw{lab04-01}{h!}{7cm}{Установка даты отчета о состоянии проекта}

После каждой установки фактических параметров производим выравнивание ресурсов
для предотвращения перегрузки.

С помощью окна \texttt{Обновление задач}, вызывающегося с помощью
меню~\texttt{Задача -> Планирование -> По графику -> Обновить задачи}, задаем
даты завершение работ 6 и 9~(рисунки~\ref{img:lab04-02}-\ref{img:lab04-03}).

\imgw{lab04-02}{h!}{10cm}{Установка фактической даты окончания работы 6}

\imgw{lab04-03}{h!}{10cm}{Установка фактической даты окончания работы 9}

Задаем процент завершения работы 17 с помощью того же окна~(рисунок~\ref{img:lab04-04}).

\imgw{lab04-04}{h!}{10cm}{Установка процента завершения работы 17}

Учитываем курсы повышение квалификации ведущего программиста с помощью
установки доступности ресурса~(рисунок~\ref{img:lab04-05}), а также повышение
зарплаты после этого с помощью изменения таблицы норм затрат по
умолчанию~(рисунок~\ref{img:lab04-06}).

\imgw{lab04-05}{h!}{10cm}{Учет курсов повышения квалификации ведущего программиста}

\imgw{lab04-06}{h!}{10cm}{Учет повышения заработной платы ведущего программиста после курсов}

На время курсов повышения квалификации попадают два
совещания~(рисунок~\ref{img:lab04-07}), будем считать, что ведущий программист
данные совещания не посещает. Также после введенных изменений проект стал
оканчиваться позже~(28.07.2023), необходимо учесть совещания в добавленный
промежуток премени. Последний текущий параметр по посещению совещаний с
10~апреля только в случае выполнения работы и за две недели до этого также
должен быть учтен.

\imgw{lab04-07}{h!}{10cm}{Совещания, которые ведущий программист не должен посещать}

После выполнения последнего условия стало ясно, что последнее совещание
посещает только ведущий программист, а следовательно не имеет смысла их
проводить.
