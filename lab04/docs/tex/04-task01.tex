\clearpage
\section{Актуализация параметров проекта}

\subsection{Текущие параметры проекта}

Дата отчета: 17.04.2023

На эту дату завершились все работы, которые должны были завершиться, кроме:

\begin{itemize}
    \item работа 6 завершилась на неделю позже;
    \item работа 9 завершилась на 10 дней позже;
    \item работа 17 выполнена на 10\%;
    \item с 1~апреля на 2~недели ведущий программист уехал на курсы повышения
	  квалификации, после возвращения с которых его заработная плата
	  поднята на~10\%;
    \item с 10~апреля совещания посещают только те сотрудники, которые
        выполняют работы на данный момент или будут выполнять работу через
        2~недели и меньше; после окончания своих работ сотрудники совещания не
        посещают.
\end{itemize}

\subsection{Настройка параметров проекта}

Задаем дату отчета во вкладке \texttt{Проект -> Состояние -> Дата отчета о
состоянии}, что представлено на рисунке~\ref{img:lab04-01}.

\imgw{lab04-01}{h!}{7cm}{Установка даты отчета о состоянии проекта}

С помощью окна \texttt{Обновление задач}, вызывающегося с помощью
меню~\texttt{Задача -> Планирование -> По графику -> Обновить задачи}, задаем
даты завершение работ 6 и 9~(рисунки~\ref{img:lab04-02}-\ref{img:lab04-03}).

\imgw{lab04-02}{h!}{10cm}{Установка фактической даты окончания работы 6}

\imgw{lab04-03}{h!}{10cm}{Установка фактической даты окончания работы 9}

Задаем процент завершения работы 17 с помощью того же
окна~(рисунок~\ref{img:lab04-04}).

\imgw{lab04-04}{h!}{10cm}{Установка процента завершения работы 17}

Учитываем курсы повышение квалификации ведущего программиста с помощью
установки доступности ресурса~(рисунок~\ref{img:lab04-05}), а также повышение
зарплаты после этого с помощью изменения таблицы норм затрат по
умолчанию~(рисунок~\ref{img:lab04-06}).

\imgw{lab04-05}{h!}{10cm}{Учет курсов повышения квалификации ведущего программиста}

\imgw{lab04-06}{h!}{10cm}{Учет повышения заработной платы ведущего программиста
после курсов}

После проведенных изменений необходимо провести выравнивание для устранения
возникших перегрузок ресурсов, которое влечет изменение дат начала и конца
некоторых задач, что может повлиять на посещаемые сотрудниками совещания.  Здесь
же сразу учтем, что ведущий программист не посещал совещания~5~и~6 во время
курсов повышения квалификации, и установим процент выполнения остальных задач по
графику.

Результат учета всех кроме последнего фактических параметров приведен на
рисунке~\ref{img:lab04-07}.

\imgw{lab04-07}{h!}{17cm}{Влияние фактических параметров кроме последнего}

Дата окончания проекта сдвинулась на 9~дней~(28.07.2023), затраты увеличились,
составив 49~285~рублей, что все еще укладывается в бюджет, но его запас сокращен
на 776~рублей.

Учтем последний фактический параметр по совещаниям, что позволит сократить
затраты.

Хотя после введенных изменений проект стал оканчиваться 9~дней
позже~(28.07.2023) и должно быть проведено еще одно совещания, по правилам,
введенных с 10~апреля данное совещание должен будет посетить один только ведущий
программист, в чем нет необходимости. 

По последнему фактическому параметру исключим ресурсы с совещаний, которые они
не должны посещать. Посещаемые совещания представлены на
рисунке~\ref{img:lab04-08}.

\imgw{lab04-08}{h!}{17cm}{Представление ресурсов с учетом последнего
фактического параметра}

После учета правил посещения совещаний, вступивших в силу с 10~апреля, дата
окончания проекта осталась такой же, как и без учета, однако при этом удалось
сократить затраты до 48~700~рублей, что на 190~рублей больше затрат по базовому
плану. Диаграмма Ганта проекта с учетом фактических параметров и линией
прогресса представлена на рисунке~\ref{img:lab04-09}. Отклонение от базового
плана представлено на рисунке~\ref{img:lab04-10}.

\imgw{lab04-09}{h!}{17cm}{Диаграмма Ганта проекта с учетом фактических
параметров}

\imgw{lab04-10}{h!}{17cm}{Отклонения от базового плана}

Таким образом, отклонение от базового плана составило 9~дней в сторону
увеличения длительности проекта и 190 рублей затрат также в сторону увеличения.
