\section{Метод функциональных точек}

\subsection{Описание}

\textbf{Функциональная точка} --- это единица измерения функциональности программного
обеспечения.

\textbf{Пользователи} --- это отправители и целевые получатели данных, ими могут быть как
реальные люди, так и смежные интегрированные информационные системы.

\textbf{Функциональные типы} — логических группы взаимосвязанных данных, используемых и поддерживаемых приложением:

\begin{itemize}
    \item внешний ввод (EI) --- транзакция получения данных от пользователя;
    \item внешний вывод (ЕО) ---  транзакция перечи данных пользователю;
    \item внешний запрос (EQ) ---  интерактивный диалог с пользователем, требующий от него каких-либо действий и не связанный с вычислением производных данных или обновлением внутренних логических файлов (базы данных);
    \item внутренний логический файл (ILF) --- информация, которая используется во внутренних взаимодействиях системы;
    \item внешний интерфейсный файл (EIF) ---файлы, участвующие во внешних взаимодействиях с другими системами.
\end{itemize}

Для оценки сложности функциональных типов используются следующие характеристики (их количество):

\begin{itemize}
    \item DET (data element type) --- это уникальное распознаваемое пользователем, неповторяющееся поле данных;
    \item RET (record element type) --- идентифицируемая пользователем логическая группа данных внутри ILF или EIF;
    \item FTR (file type referenced) --- это тип файла, на который ссылается транзакция.
\end{itemize}

Оценка числа функциональных точек происходит по алгоритму:

\begin{enumerate}
    \item определение функциональных типов приложения и их количества;
    \item определение количества связанных с каждым функциональным типом элементарных данных (DET), элементарных записей (RET) и файлов типа ссылок (FTR)
    \item оценка сложности каждого типа по соответствующим таблицам и определение количества функциональных типов каждой сложности;
    \item подсчет количества функциональных точек приложения, путем умножения числа количества функциональных типов каждой сложности на соответстсвующие коэффициенты и сложения полученных результатов;
    \item проведение корректировки числа функциональных точек  с учетом общих характеристик системы в соответствии с формулой:

    $$\text{AFP} = \text{FP} \cdot (0.65 + 0.01 \cdot \sum\limits_{i = 1}^{14} F_i),$$

    где $\text{FP}$ --- число функциональных точек, полученное на предыдущем шаге,

    $F_i$ --- $i$-ый коэффициент регулировки сложности, принимающий значения от 0~до~5~(нет влияния, случайное, небольшое, среднее, важное, основное), которое выбирается по соответствующей таблице.
\end{enumerate}

Из числа функциональных точек выводится количество строк кода по таблицам соответствия их числа (число строк кода на одну функциональную точку), построенных на основе статистических данных в зависимости от языка программирования.

\subsection{Применение}

\subsubsection{Постановка задачи}

Компания получила заказ на разработку клиентского мобильного приложения брокерской системы. Программа позволяет просматривать актуальную биржевую информацию, производить сделки и отслеживать их выполнение. Приложение имеет 4 страницы: авторизация, биржевые сводки, заявки, новая заявка.

\subsubsection{Страница <<Авторизация>>}

На данной странице осуществляется ввод логина и пароля пользователя для входа в систему.
Страница содержит \textbf{два поля ввода} и \textbf{одну командную кнопку}, а также \textbf{флажок-переключатель}, который активируется при необходимости запоминания параметров авторизации.

По данной странице можно выделить следующий набор функциональных типов:

\begin{itemize}
    \item внутренние логические файлы:
        \begin{itemize}
            \item таблица пользователей бд с полями логина и пароля:

            $\text{DET} = 2$ --- логин, пароль;\\
            $\text{RET} = 1$ --- логин и пароль являются строками;\\
            Уровень:  низкий;
            
            \item локальный файл для хранения данных одного пользователя:

            $\text{DET} = 2$ --- логин, пароль;\\
            $\text{RET} = 1$ --- логин и пароль являются строками;\\
            Уровень:  низкий;
            
        \end{itemize}
    \item внешник логический файл: отсутствует;
    \item внешний ввод: запоминание данных пользователя:
    
        $\text{DET} = 4$ --- логин, пароль, флажок, кнопка;\\
        $\text{FTR} = 1$ --- локальный файл для данных одного пользователя.\\
        Уровень: низкий;

    \item внешний вывод: отсутствует;
    \item внешний запрос: авторизация:

        $\text{DET} = 4$ --- логин, пароль, флажок, кнопка;\\
        $\text{FTR} = 1$ --- локальный файл для данных одного пользователя.\\
        Уровень: низкий.
\end{itemize}

Итого по странице:

\begin{itemize}
    \item 2 ILF низкого уровня;
    \item 1 EI низкого уровня;
    \item 1 EQ низкого уровня.
\end{itemize}

\subsubsection{Страница <<Биржевые сводки>>}

Страница содержит \textbf{таблицу}, \textbf{кнопку} «Добавить» и диалоговое окно с \textbf{одним полем для ввода} и \textbf{двумя командными кнопками}.

Таблица содержит три колонки: Ценная бумага (имя бумаги), Цена (цена за одну ценную бумагу), Изменение (изменение цены бумаги со времени последнего закрытия биржи). Кнопка «Добавить» вызывает \textbf{диалоговое окно} для добавления новой бумаги (окно состоит из поля ввода и кнопок ОК, Cancel).

По данной странице можно выделить следующий набор функциональных типов:

\begin{itemize}
    \item внутренний логический файл: таблица ценных бумаг с полем имени:
    
        $\text{DET} = 1$ --- имя;\\
        $\text{RET} = 1$ --- имя --- строка;\\
        Уровень: низкий.
        
    \item внешний логический файл: информация по цене и изменению о ценных бумагах;

        $\text{DET} = 3$ --- имя, цена, изменение;\\
        $\text{RET} = 1$ --- имя --- строка, цена и изменение --- вещественный тип;\\
        Уровень: низкий.
        
    \item внешний ввод: добавление новой бумаги:
    
        $\text{DET} = 3$ --- текстовое поле для имени, кнопка <<Cancel>>, кнопка <<OK>>; \\
        $\text{FTR} = 1$ --- добавление записи во внутренний логический файл.\\
        Уровень: низкий.
        
    \item внешний вывод информации по ценным бумагам:
    
        $\text{DET} = 3$ --- имя, цена, изменение; \\
        $\text{FTR} = 2$ --- обращение к внутреннему файлу с именами бумаг и к внешнему с информацией о цене и изменении;\\
        Уровень: низкий;
        
    \item внешний запрос: отсутсвует.
\end{itemize}

Итого по странице:

\begin{itemize}
    \item 1 ILF низкого уровня;
    \item 1 EIF низкого уровня;
    \item 1 EI низкого уровня;
    \item 1 EO низкого уровня.
\end{itemize}

\subsubsection{Страница <<Заявки>>}

Заявки содержат \textbf{таблицу}, отображающую текущие (еще не выполненные) заявки на покупку или продажу ценных бумаг. Таблица содержит четыре поля: Тип (покупка/продажа), Имя бумаги, Цена, по которой готовы покупаться/продаваться бумаги, Количество бумаг для покупки/продажи. При нажатии на любую строку таблицы появляется контекстное меню с возможностью \textbf{удалить} или \textbf{изменить заявку}.

По данной странице можно выделить следующий набор функциональных типов:

\begin{itemize}
    \item внутренний логический файл: таблица текущих заявок:

        $\text{DET} = 4$ --- тип, имя, цена, количество;\\
        $\text{RET} = 4$ --- тип --- логический, имя --- строка, цена --- вещественное число; количество --- целое число;\\
        Уровень: низкий.
    
    \item внешний логический файл: отсутствует
    \item внешний ввод:
        \begin{itemize}
            \item удаление:
            
                $\text{DET} = 5$ --- тип, имя, цена, количество, кнопка; \\
                $\text{FTR} = 1$ --- обращение к внутреннему логическому файлу;\\
                Уровень: низкий;
                
            \item изменение:

                $\text{DET} = 5$ --- тип, имя, цена, количество, кнопка; \\
                $\text{FTR} = 1$ --- обращение к внутреннему логическому файлу;\\
                Уровень: низкий;
                
        \end{itemize}
    \item внешний вывод информации о заявках:

        $\text{DET} = 4$ --- тип, имя, цена, количество; \\
        $\text{FTR} = 1$ --- обращение к внутреннему логическому файлу;\\
        Уровень: низкий;
    
    \item внешний запрос: отсутсвует.
\end{itemize}

\begin{itemize}
    \item 1 ILF низкого уровня;
    \item 2 EI низкого уровня;
    \item 1 EO низкого уровня.
\end{itemize}

\subsubsection{Страница <<Новая заявка>>}

Страница позволяет оформить заявку на покупку или продажу ценной бумаги. Страница состоит из \textbf{4 полей}: Бумага (имя бумаги), Цена (цена, по которой необходимо купить/продать бумагу), Покупка (булева переменная в значение true обозначает покупку, false – продажа) и кнопки «Оформить» - для подтверждения оформления заявки.

По данной странице можно выделить следующий набор функциональных типов:

\begin{itemize}
    \item внутренний логический файл: тот же, что и на странице заявок;
    \item внешний логический файл: отсутствует;
    \item внешний ввод: создание новой заявки:

        $\text{DET} = 5$ --- поля ввода имя, цена, количество, флаг покупки и кнопка; \\
        $\text{FTR} = 1$ --- обращение к внутреннему логическому файлу;\\
        Уровень: низкий;
    
    \item внешний вывод: отсутствует;
    \item внешний запрос: отсутствует.
\end{itemize}

\begin{itemize}
    \item 1 EI низкого уровня.
\end{itemize}

\subsection{Расчеты}

Всего функциональных типов:

\begin{itemize}
    \item 5 EI низкого уровня;
    \item 2 EO низкого уровня;
    \item 1 EQ низкого уровня;
    \item 4 ILF низкого уровня;
    \item 1 EIF низкого уровня.
\end{itemize}

Для уточнения числа функциональных точек используются следующие значения характеристики продукта:

\begin{itemize}
    \item Обмен данными --- 5
    \item Распределенная обработка --- 5
    \item Производительность --- 3
    \item Эксплуатационные ограничения по аппаратным ресурсам --- 2
    \item Транзакционная нагрузка --- 3
    \item Интенсивность взаимодействия с пользователем (оперативный ввод данных) --- 4
    \item Эргономические характеристики, влияющие на эффективность работы конечных пользователей --- 1
    \item Оперативное обновление --- 4
    \item Сложность обработки --- 4
    \item Повторное использование --- 0
    \item Легкость инсталляции --- 1
    \item Легкость эксплуатации/администрирования --- 2
    \item Портируемость --- 2
    \item Гибкость --- 2 
\end{itemize}

Результаты расчетов числа функциональных точек представлены на рисунке~\ref{img:fp01}.

\imgw{fp01}{h!}{14cm}{Результаты расчетов по методу функциональных точек}

Таким образом, число функциональных точек:

\begin{itemize}
    \item первоначальное = 59;
    \item скорректированное = 60.77.
\end{itemize}

Процентное соотношения языков программирования:

\begin{itemize}
    \item SQL --- 15\%;
    \item C\# --- 60\%;
    \item Java --- 25\%.
\end{itemize}

С учетом данного соотношения с использованием таблиц соответствия числа функциональных точек строкам кода получаем, что проект будет состоять по данной оценке из 2856~строк кода.