\section{Модель композиции приложения в COCOMO II}

\subsection{Описание модели}

Для оценки стоимости проекта на ранней стадии конструирования (создание прототипов, макетирование пользовательских интерфейсов, анализ осуществимости, оценка производительности, определение степени зрелости технологии) используется модель композиции приложения. Для такой модели характерны грубые входные данные, оценки низкой точности, приблизительные требования и архитектура на уровне концепции. 

В качестве единицы измерения используется объектная точка --- средство косвенного измерения программного обеспечения. Определение числа объектных точек производится с учетом количества экранов (как элементов пользовательского интерфейса), отчетов и компонентов, требуемых для построения приложения, в соответствии со следующими правилами:

\begin{itemize}
    \item простые экранные формы принимаются за одну объектную точку, умеренной сложности --- за две объектные точки, сложные --- за три объектные точки;
    \item простые отчеты принимаются за две объектные точки, умеренной сложности --- за пять объектных точек, сложные --- за восемь объектных точек;
    \item модули, написанные на языках программирования третьего поколения, считаются за десять объектных точек.
\end{itemize}

Новые объектные точки NOP определяются по следующей формуле:

$$\text{NOP} = \text{Объектные точки} \cdot \frac{(100 - \%\text{RUSE})}{100},$$

\noindentгде \%$\text{RUSE}$ --- процент повторного использования кода программы.

Трудозатраты вычисляются так:

$$\text{Трудозатраты} = \frac{\text{NOP}}{\text{PROD}},$$

\noindentгде $\text{PROD}$ --- оценка скорости разработки, определяемая из опытности или возможности разработчика или команды или из зрелости или возможности среды разработки.

Время разработки получается в соответствии с формулой:

$$\text{Время} = 3 \cdot \text{Трудозатраты}^{0.33 + 0.2 \cdot (\text{p} - 1.01)},$$

\noindentгде $\text{p}$ --- показатель степени. Значение показателя степени рассчитывается с учетом факторов, влияющих на показатель степени:

$$\text{p} = \frac{(\text{PREC} + \text{FLEX} + \text{RESL} + \text{TEAM} + \text{PMAT})}{100} + 1.01.$$

\subsection{Применение}

Из макета интерфейса:

\begin{itemize}
    \item для страницы <<Авторизация>>:
        \begin{itemize}
            \item одна форма средней сложности;
        \end{itemize}
    \item для страницы <<Биржевые сводки>>:
        \begin{itemize}
            \item одна сложная форма (таблица биржевых сводок);
            \item одна форма средней сложности (форма ввода);
        \end{itemize}
    \item для страницы <<Заявки>>:
        \begin{itemize}
            \item одна форма средней сложности (таблица заявок);
            \item одна простая форма (кнопки <<Удалить>> и <<Изменить>>);
        \end{itemize}
    \item для страницы <<Новая заявка>>:
        \begin{itemize}
            \item одна форма средней сложности (форма добавления заявки).
        \end{itemize}
\end{itemize}

Итого:

\begin{itemize}
    \item 1 простая форма;
    \item 4 формы средней сложности;
    \item 1 сложная форма.
\end{itemize}

Разработанное приложение состоит из трех компонентов: первый компонент написан
на SQL, второй --- на С\#, а третий --- на Java. Поэтому в проекте два модуля,
написанных на языках третьего поколения (C\# и Java).

Из условия задания:

\begin{itemize}
    \item \%RUSE = 0 \%;
    \item опытность команды --- низкая.
\end{itemize}

\subsubsection{Факторы, влияющие на показатель степени}

Из условия задания:

\begin{itemize}
    \item у отдельных членов команды имеется некоторый опыт создания систем подобного типа, поэтому новизна проекта (PREC) --- наличие некоторого количества прецедентов;
    \item заказчик не настаивает на жесткой регламентации процесса, однако график реализации проекта довольно жесткий, поэтому гибкость процесса разработки (FLEX) --- большей частью согласованный процесс;
    \item анализу архитектурных рисков было уделено лишь некоторое внимание, поэтому разрешение рисков в архитектуре системы (RESL) --- некоторое (40 \%);
    \item в целях сплочения команды были проведены определенные мероприятия, что обеспечило на старте проекта приемлемую коммуникацию внутри коллектива, поэтому сплоченность команды (TEAM) --- некоторая согласованность;
    \item организация только начинает внедрять методы управления проектами и формальные методы оценки качества процесса разработки, поэтому уровень зрелости процесса разработки (PMAT) --- уровень 1 СММ.
\end{itemize}

Тогда имеем:

$$\text{p} = \frac{(3.72 + 2.03 + 5.65 + 3.29 + 7)}{100} + 1.01 = 1.2269.$$

\subsection{Результат}

На рисунке~\ref{img:app-composition} показана оценка трудозатрат и длительности разработки с использованием модели композиции приложения.

\imgw{app-composition}{h!}{12cm}{Результаты расчетов по модели композиции приложения}

Средняя численность команды определяется по следующей формуле:

$$\text{Численность команды} = \frac{\text{Трудозатраты}}{\text{Время}} =
\frac{4.57}{5.29} = 1 \text{ работник.}$$

Предварительная оценка бюджета для средней зарплаты 70 000 рублей проводится по
следующей формуле:

$$\text{Бюджет} = \text{Трудозатраты} \cdot \text{Средняя зарплата} = 4.57 \cdot
70 000 = 320 000\text{ рублей.}$$
