\section{Модель ранней разработки архитектуры в COCOMO II}

\subsection{Описание модели}

Для получения приблизительных оценок проектных затрат периода выполнения проекта перед тем, как будет определена архитектура в целом, применяется модель ранней разработки архитектуры. Для такой модели характерны  оценки умеренной точности и ясно понимаемые особенности проекта, требования и архитектура. В качестве единиц измерения используются функциональные точки или KSLOC. 

Трудозатраты вычисляются так:

$$\text{Трудозатраты} = 2.45 \cdot \text{EArch} \cdot \text{Размер}^p,$$

\noindentгде $\text{Размер} = \text{KSLOC}$, $\text{EArch}$ определяется через произведение множителей трудоемкости:
$$\text{EArch} = \text{PERS} \cdot \text{RCPX} \cdot \text{RUSE} \cdot \text{PDIF} \cdot \text{PREX} \cdot \text{FCIL} \cdot \text{SCED}.$$

Время разработки получается в соответствии с формулой:

$$\text{Время} = 3 \cdot \text{Трудозатраты}^{0.33 + 0.2 \cdot (\text{p} - 1.01)},$$

\noindentгде $\text{p}$ --- показатель степени. Значение показателя степени рассчитывается с учетом факторов, влияющих на показатель степени:

$$\text{p} = \frac{(\text{PREC} + \text{FLEX} + \text{RESL} + \text{TEAM} + \text{PMAT})}{100} + 1.01.$$

\subsection{Применение}

Из условия задания:

\begin{itemize}
    \item надежность и сложность продукта (RCPX) --- очень высокие;
    \item повторное использование компонентов (RUSE) --- низкий;
    \item опытность персонала (PERS) --- низкая;
    \item способности персонала (PREX) --- номинальные;
    \item сложность платформы (PDIF) --- высокая;
    \item возможности среды (FCIL) --- очень высокие;
    \item сроки (SCED) --- очень высокие.
\end{itemize}

\subsection{Результат}

На рисунке~\ref{img:early-architecture} представлена оценка трудозатрат и длительности разработки с использованием модели ранней разработки архитектуры.

\imgw{early-architecture}{h!}{12cm}{Результаты расчетов по модели ранней разработки архитектуры}

Средняя численность команды определяется по следующей формуле:

$$\text{Численность команды} = \frac{\text{Трудозатраты}}{\text{Время}} = \frac{19.2}{9.03} = 3 \text{ работника.}$$

Предварительная оценка бюджета для средней зарплаты 70 000 рублей проводится по следующей формуле:

$$\text{Бюджет} = \text{Трудозатраты} \cdot \text{Средняя зарплата} = 19.12 \cdot 70 000 = 1 338 054.96 \text{ рублей.}$$