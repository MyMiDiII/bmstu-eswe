\section{Задание 3: Оптимизация критического пути }

Критический путь показан на рисунке~\ref{img:lab03-task03-01}. Самыми
длительными задачами являются те, на которые назначены программисты. Рассмотрим
возможность уменьшения их длительнсти с помощью назначения на них
дополнительный ресурсов <<Программист>>, которые ранее не привлекались к данным
задачам~(рисунок~\ref{img:lab03-task03-02}).

\imgw{lab03-task03-01}{h!}{17cm}{Критический путь}

\imgw{lab03-task03-02}{h!}{17cm}{Задачи, выполняемые программистами}

Назначим всех программистов на все задачи, которые они могут выполнять, и
проведем выравнивание. Удалим совещания после сдачи сайта в эксплуатацию.
Получим результат, приведенный на рисунке~\ref{img:lab03-task03-03}.
Распределение задач по программистам представлено на
рисунке~\ref{img:lab03-task03-04}.

\imgw{lab03-task03-04}{h!}{17cm}{Задачи, выполняемые программистами, после оптимизации}

\imgw{lab03-task03-03}{h!}{17cm}{Длительность задач, окончание проекта и
затраты на него}

Таким образом, после проведения оптимизации, по плану выполнение проекта
окончится 17.07.2023, что на 1.5~месяца раньше запланированного срока, также
затраты уменьшатся до 48~342~рублей за счет сокращения длительности задач с
помощью увеличения количества ресурсов.

Приведем диаграммы затрат и трудозатрат до~(рисунок~\ref{img:lab02-task03})
и после~(рисунок~\ref{img:lab03-task03}) оптимизации.
По данным диаграммам видно, что соотношение затрат и трудозатрат не изменилось,
это связано с тем, что мы сократили длительность задач за счет привлечения к
ним дополнительных ресурсов при фиксированных трудозатратах.

\imgw{lab02-task03}{h!}{17cm}{Соотношения затрат и трудозатрат до оптимизации}

\imgw{lab03-task03}{h!}{17cm}{Соотношения затрат и трудозатрат после оптимизации}

Полученные параметры проекта соответствуют требованиям, поэтому сохраним
базовый план~(рисунок~\ref{img:lab03-total}).

\imgw{lab03-total}{h!}{10cm}{Сохранение базового плана}
