\section{Информация о проекте}

Команда разработчиков из 16~человек занимается созданием карты города на основе
собственного модуля отображения. Проект должен быть завершен в течение
6~месяцев. Бюджет проекта:~50~000~рублей.

По итогам предыдущих лабораторных работ для проекта были получены затраты в
размере 48~126~рублей, а дата окончания --- 18.09.2023.

Ресурсный лист, полученный в ходе выполнения лаборатной работы~№2, представлен
на рисунке~\ref{img:lab02-task02-01}.

\imgw{lab02-task02-01}{h!}{17cm}{Ресурсный лист с указанием перегруженных
ресурсов}

Как было выявнено в предыдущей работе, перегрузки <<Системного аналитика>>,
<<Художника-дизайнера>> и <<Технического писателя>> возникают в силу
необходимости одновременного выполнения каждым из них нескольких задач.

Способами устранения перегрузки ресурсов являются:
\begin{itemize}
    \item изменение календаря работы ресурса;
    \item назначение ресурса на неполный рабочий день;
    \item изменение профиля назначения ресурса;
    \item изменение ставки оплаты ресурса;
    \item добавления ресурсу времени задержки;
    \item выделения этапов в задаче и перекрытие по времени их выполнения;
    \item применение автоматического выравнивания.
\end{itemize}

\section{Задание 1: Выравнивание загрузки ресурсов в проекте}

В силу того, что в проекте три перегруженных ресурса, используется
автоматическое выравнивание для того, чтобы MS Project автоматически выбрал
лучший вариант решения перегрузок с учетом связи задач.

Параметры автоматического выравнивания приведены на
рисунке~\ref{img:lab03-task01-01}.

\imgw{lab03-task01-01}{h!}{12cm}{Параметры автоматического выравнивания}

Результат автоматического выравнивания приведены на
рисунке~\ref{img:lab03-task01-02}.

\imgw{lab03-task01-02}{h!}{17cm}{Результат автоматического выравнивания}

Как видно по столбцу индикаторов, перегруженные ресурсы после выравнивания
отсутствуют. Разгрузка <<Системного аналитика>> и <<Технического писателя>>
произошла за счет сдвига начала одной из одновременно выполняемых задач на
конец другой. В силу сдвига дат выполнения вехи по <<Построению базы объектов>>
удалось сократить количество праздничных и выходных дней, а следовательно,
уменьшить время работы <<Сервера>>, что позволило сократить затраты, которые
теперь составляют~---~48~076~рублей. Разгрузка <<Художника дизайнера>>
произведена за счет задержки оконачания работы над задачей на 2~дня, в силу
чего произошел сдвиг всех последующих задач и, как следствие, увеличение даты
окончания проекта также на 2~дня~---~20.09.2023.
