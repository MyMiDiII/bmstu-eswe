\section{Задание 2: Учет периодических задач в плане проекта}

Добавление периодической задачи <<Совещание>> происходит через окно
\texttt{Сведения о повторяющейся задаче}. Ввод параметров данной задачи согласно
условию приведены на рисунке~\ref{img:lab03-task02-01}.

\imgs{lab03-task02-01}{h!}{0.8}{Сведения о повторяющейся задаче <<Совещание>>}

Назначение всех специалистов кроме программистов и наборщиков данных на
<<Совещание>> представлено на рисунке~\ref{img:lab03-task02-02}.

\imgs{lab03-task02-02}{h!}{0.8}{Назначение ресурсов на <<Совещание>>}

Результаты добавления <<Совещания>> приведены на
рисунке~\ref{img:lab03-task02-03}, из которого видно, что снова появились
перегруженные ресурсы и возникли значительные затраты на новую задачу в
$\sim$18~523~рубля.

\imgw{lab03-task02-03}{h!}{17cm}{Диграмма Ганта проекта после добавления <<Совещания>>}

Из представления \texttt{Лист ресурсов}~(рисунок~\ref{img:lab03-task02-04})
видно, что возникла перегрузка тех ресурсов, которые участвуют в <<Совещании>>,
причина этого --- наложение времени совещаний на время выполнения основных
задач~(рисунок~\ref{img:lab03-task02-05}).

\imgw{lab03-task02-04}{h!}{17cm}{Лист ресурсов после добавления <<Совещания>>}

\imgw{lab03-task02-05}{h!}{12cm}{Визуализация наложения задач некоторых ресурсов}
