\section{Задание 2: Учет периодических задач в плане проекта}

Добавление периодической задачи <<Совещание>> происходит через окно
\texttt{Сведения о повторяющейся задаче}. Ввод параметров данной задачи согласно
условию приведены на рисунке~\ref{img:lab03-task02-01}.

\imgs{lab03-task02-01}{h!}{0.8}{Сведения о повторяющейся задаче <<Совещание>>}

Назначение всех специалистов кроме программистов и наборщиков данных на
<<Совещание>> представлено на рисунке~\ref{img:lab03-task02-02}.

\imgs{lab03-task02-02}{h!}{0.8}{Назначение ресурсов на <<Совещание>>}

Результаты добавления <<Совещания>> приведены на
рисунке~\ref{img:lab03-task02-03}, из которого видно, что снова появились
перегруженные ресурсы и возникли значительные затраты на новую задачу в
$\sim$20~039~рублей.

\imgw{lab03-task02-03}{h!}{17cm}{Диграмма Ганта проекта после добавления <<Совещания>>}

Из представления \texttt{Лист ресурсов}~(рисунок~\ref{img:lab03-task02-04})
видно, что возникла перегрузка тех ресурсов, которые участвуют в <<Совещании>>,
причина этого --- наложение времени совещаний на время выполнения основных
задач~(рисунок~\ref{img:lab03-task02-05}).

\imgw{lab03-task02-04}{h!}{17cm}{Лист ресурсов после добавления <<Совещания>>}

\imgw{lab03-task02-05}{h!}{12cm}{Визуализация наложения задач некоторых ресурсов}

Для устранения перегрузки ресурсов проведем автоматическое выравнивание еще
раз.  После выравнивания в задачи были добавлены <<перерывы>> на совещание.
Дата окончания проекта стала 25.09.2023~(рисунок~\ref{img:lab03-task02-06}).

\imgw{lab03-task02-06}{h!}{17cm}{Результат выравнивания после введения совещаний}

После введения совещаний затраты на них составляют около трети всех затрат,
также сроки проекта превышаются на 25~дней. Оптимизация временный параметров
будет проведена в задании 3. Проведем оптимизацию затрат.

Во время совещаний сотрудники не занимают свои рабочие места, поэтому можно
исключить из каждого используемого на совещании трудового ресурса затраты на
использование и ввести фиксированные затраты на совещания, которые будут
меньше.

Исключение затрат на использование проводится путем добавление таблицы норм
затрат~B к каждому ресурсу~(рисунок~\ref{img:lab03-task02-07}). Указание для
каждого ресурса каждого совещания таблицы норм затрат~B производится через
одноименный столбец представления \texttt{Использование
задач}~(рисунок~\ref{img:lab03-task02-08}).

\imgw{lab03-task02-07}{h!}{12cm}{Добавление новой таблицы норм затрат}

\imgw{lab03-task02-08}{h!}{15cm}{Установка таблицы норм затрат ресурсу по
конретной задаче}

Итого получим затраты в размере~49~849~рублей, что укладывается в
бюджет~(рисунок~\ref{img:lab03-task02-09}).

\imgw{lab03-task02-09}{h!}{17cm}{Результат оптимизации финансовых параметров}
