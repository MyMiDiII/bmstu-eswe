\section{Модель оценки стоимости COCOMO}

\textbf{COnstructive COst MOdel} -- алгоритмическая модель оценки стоимости разработки программного обеспечения. Она использует простую формулу регрессии с параметрами, определенными из данных, собранных по ряду проектов.

\begin{equation}
    \text{Трудозатраты} = \text{С1} \cdot \text{EAF} \cdot (\text{Размер})^{\text{p1}}
\end{equation}

\begin{equation}
    \text{Время} = \text{С2} \cdot (\text{Трудозатраты})^{\text{p2}}
\end{equation}

\noindent где:

\begin{itemize}
    \item \texttt{трудозатраты} --- количество человеко-месяцев;
    \item \texttt{С1} --- масштабирующий коэффициент;
    \item \texttt{EAF} --- уточняющий фактор, характеризующий предметную область,
    персонал, среду и инструментарий, используемый для создания рабочих
    продуктов процесса;
    \item \texttt{размер} --- размер конечного продукта (кода, созданного
    человеком), измеряемый в исходных инструкциях (DSI, delivered source
    instructions), которые необходимы для реализации требуемой функциональной
    возможности;
    \item \texttt{P1} --- показатель степени, характеризующий экономию при
        больших масштабах, присущую тому процессу, который используется для
        создания конечного продукта; в частности, способность процесса избегать
        непроизводительных видов деятельности (доработок, бюрократических
        проволочек, накладных расходов на взаимодействие);
    \item \texttt{время} --- общее количество месяцев;
    \item \texttt{С2} --- масштабирующий коэффициент для сроков исполнения;
    \item \texttt{Р2} --- показатель степени, который характеризует инерцию и
    распараллеливание, присущие управлению разработкой ПО.
\end{itemize}

Коэффициенты \texttt{C1}, \texttt{P1}, \texttt{C2}, \texttt{P2} зависят от
режима модели COCOMO, а коэффициент \texttt{EAF} является произведением
коэффициентов драйверов, зависящих от их уровня и представленных на
рисунке~\ref{img:cocomo1}.

\imgw{cocomo1}{h!}{17cm}{Значения драйверов затрат в модели COCOMO}
