\section{Задание 1}

\subsection{Условие}

Исследовать влияние атрибутов программного продукта (\textbf{RELY}, \textbf{DATA} и \textbf{CPLX}) на трудоемкость (\textbf{РМ}) и время разработки (\textbf{ТМ}) для модели \textbf{COCOMO} и промежуточного типа проекта. Для этого получить значения PM и ТМ для одного и того же значения размера программного кода (\textbf{SIZE}), изменяя значения указанных драйверов от очень низких до очень высоких. Сначала провести анализ при отсутствии ограничений на сроки разработки, выбрав номинальное значение параметра \textbf{SCED}. 

Какой из трех указанных драйверов затрат оказывает большее влияние на сроки реализации проекта и объем работ? Проанализировать, как изменятся значения \textbf{PM} и \textbf{ТМ} при наличии более жестких ограничений на сроки разработки (драйвер \textbf{SCED} изменяется от высокого до очень высокого). Результаты исследований оформить графически и сделать соответствующие выводы.

\subsection{Результаты}

На рисунках \ref{img:task1-nominal}-\ref{img:task1-veryhigh} приведены результаты графических исследований атрибутов (\textbf{RELY}, \textbf{DATA} и \textbf{CPLX}) при различных уровнях драйвера \textbf{SCED}.

\imgh{task1-nominal}{t!}{7cm}{Уровень SCED: номинальный}

\imgh{task1-high}{t!}{7cm}{Уровень SCED: высокий}

\imgh{task1-veryhigh}{t!}{7cm}{Уровень SCED: очень высокий}

\subsection{Вывод}

При увеличении уровней исследуемых атрибутов увеличиваются трудозатраты и время, так как происходит повышение требований к проекту.

При номинальном значении драйвера \textbf{SCED} наибольшее влияние на сроки реализации проекта и объем работ оказывает драйвер \textbf{RELY} (требуемая надежность), на что указывает угол наклона графика.

Более строгие ограничения на сроки разработки не сильно влияют на трудозатраты и время. Так, при высоком уровне трудозатраты повышаются на 2.5\%, а время --- на 2.7\% относительно номинального, а при очень высоком уровне SCED трудозатраты --- на 7\%, а время --- на 3\% относительно высокого уровня.
