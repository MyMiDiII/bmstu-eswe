\section{Задание 1}

\subsection{Условие}

Исследовать влияние атрибутов программного продукта (\texttt{RELY},
\texttt{DATA} и \texttt{CPLX}) на трудоемкость (\texttt{РМ}) и время разработки
(\texttt{ТМ}) для модели \texttt{COCOMO} и промежуточного типа проекта. Для
этого получить значения PM и ТМ для одного и того же значения размера
программного кода (\texttt{SIZE}), изменяя значения указанных драйверов от очень
низких до очень высоких. Сначала провести анализ при отсутствии ограничений на
сроки разработки, выбрав номинальное значение параметра \texttt{SCED}. 

Какой из трех указанных драйверов затрат оказывает большее влияние на сроки
реализации проекта и объем работ? Проанализировать, как изменятся значения
\texttt{PM} и \texttt{ТМ} при наличии более жестких ограничений на сроки
разработки (драйвер \texttt{SCED} изменяется от высокого до очень высокого).
Результаты исследований оформить графически и сделать соответствующие выводы.

\subsection{Результаты}

На рисунках \ref{img:task1-nominal}-\ref{img:task1-veryhigh} приведены
результаты графических исследований атрибутов (\texttt{RELY}, \texttt{DATA} и
\texttt{CPLX}) при различных уровнях драйвера \texttt{SCED}.

\imgh{task1-nominal}{t!}{7cm}{Уровень SCED: номинальный}

\imgh{task1-high}{t!}{7cm}{Уровень SCED: высокий}

\imgh{task1-veryhigh}{t!}{7cm}{Уровень SCED: очень высокий}

\subsection{Вывод}

При увеличении уровней исследуемых атрибутов увеличиваются трудозатраты и время,
так как происходит повышение требований к проекту.

При номинальном значении драйвера \texttt{SCED} наибольшее влияние на сроки
реализации проекта и объем работ оказывает драйвер \texttt{RELY} (требуемая
надежность), на что указывает угол наклона графика.

Более строгие ограничения на сроки разработки не сильно влияют на трудозатраты и
время. Так, при высоком уровне трудозатраты повышаются на 2.5\%, а время --- на
2.7\% относительно номинального, а при очень высоком уровне \texttt{SCED}
трудозатраты --- на 7\%, а время --- на 3\% относительно высокого уровня.
