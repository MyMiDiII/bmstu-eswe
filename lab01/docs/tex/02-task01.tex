\chapter{Содержание проекта}

Команда разработчиков из 16~человек занимается созданием карты города на основе
собственного модуля отображения. Проект должен быть завершен в течение
6~месяцев. Бюджет проекта:~50~000~рублей.

\chapter{Задание 1: Настройка рабочей среды проекта}

Дата начала проекта (01.03.2023 --- первый рабочий день марта) задается в окне
\texttt{Проект -> Сведения о проекте}. Полная настройка сведений о проекте в
соответствии с условием приведена на рисунке~\ref{img:task01-01}.

\imgw{task01-01}{h!}{17cm}{Настройка сведений о проекте}

Следующая настройка проекта производится в окне \texttt{Файл -> Параметры ->
Расписание}.  Задется длительность работы в неделях, объем работ в часах, а тип
работ по умолчанию --- с фиксированными трудозатратами; количество рабочих
часов в день --- 8, количество рабочих часов в неделю --- 40; начало рабочей
недели в понедельник, а финансового года --- в январе; продолжительность
рабочего дня с 9~до~18~часов~(рисунок~\ref{img:task01-02}.

\imgw{task01-02}{h!}{17cm}{Настройка параметров календаря и планирования проекта}

Настройка рабочего времени: типа календаря, а также выходных, праздничных и
сокращенных дней --- приведена на рисунке~\ref{img:task01-03}.

\imgw{task01-03}{h!}{17cm}{Настройка рабочего времени}

Отображение суммарной задачи проекта установлено с помощью флага
\texttt{Суммарная задача проекта} на вкладке \texttt{Формат диаграммы Ганта}.
Также во вкладку \texttt{Заметки} в окне \texttt{Сведения о суммарной задаче}
внесена информация об основных параметрах проекта~\ref{img:task01-04}.

\imgw{task01-04}{h!}{17cm}{}
