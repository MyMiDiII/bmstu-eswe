\chapter{Выполнение тренировочного задания}

Задание по варианту №2. При создании плана пробного проекта, использовались
настройки рабочей среды проекта, заданные в MS Project по умолчанию. В качестве
даты начала работ выбран первый рабочий день марта текущего года, то
есть~01.03.2023. Осуществлялось планирование проекта со следующими временными
характеристиками:

\textbf{здесь будет таблица}

Связи между задачами:

\begin{enumerate}
    \item Предусмотреть, что C, J и D – исходные работы проекта, которые можно начинать одновременно;
    \item Работа A следует за D, а работа I – за A;
    \item Работа H следует за I;
    \item Работа F следует за H, но не может начаться, пока не завершена С;
    \item Работа G следует за I;
    \item Работа E следует за J;
    \item Работа B следует за E.
\end{enumerate}

Результаты выполнения задания приведены на рисунке~\ref{img:gant}.
Длительность проекта --- 51 день, дата завершения работ --- 10.05.2023.

\textbf{рисунок 1}

