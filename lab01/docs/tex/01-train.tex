\section{Выполнение тренировочного задания}

Задание по варианту №2. При создании плана пробного проекта, использовались
настройки рабочей среды проекта, заданные в MS Project по умолчанию. В качестве
даты начала работ выбран первый рабочий день марта текущего года, то
есть~01.03.2023. Осуществлялось планирование проекта со следующими временными
характеристиками:

{
\captionsetup{format=hang,justification=raggedleft,
              singlelinecheck=off,width=17cm}
\begin{longtable}[Hc]{|c|c|}
\caption{Временные характеристики работ}\\
    \hline
    Название работы & Длительность (дни)\\
    \hline
    Работа A & 12\\
    \hline
    Работа B & 8\\
    \hline
    Работа C & 15\\
    \hline
    Работа D & 9\\
    \hline
    Работа E & 14\\
    \hline
    Работа F & 9\\
    \hline
    Работа G & 15\\
    \hline
    Работа H & 10\\
    \hline
    Работа I & 11\\
    \hline
    Работа J & 13\\
    \hline
\end{longtable}
}

Связи между задачами:

\begin{enumerate}
    \item Предусмотреть, что C, J и D – исходные работы проекта, которые можно начинать одновременно;
    \item Работа A следует за D, а работа I – за A;
    \item Работа H следует за I;
    \item Работа F следует за H, но не может начаться, пока не завершена С;
    \item Работа G следует за I;
    \item Работа E следует за J;
    \item Работа B следует за E.
\end{enumerate}

Результаты выполнения задания приведены на рисунке~\ref{img:lab01-test}.
Длительность проекта --- 51 день, дата завершения работ --- 10.05.2023.

\imgw{lab01-test}{h!}{17cm}{Диаграмма Ганта проекта тренировочного задания}

\section{Содержание проекта}

Команда разработчиков из 16~человек занимается созданием карты города на основе
собственного модуля отображения. Проект должен быть завершен в течение
6~месяцев. Бюджет проекта:~50~000~рублей.
