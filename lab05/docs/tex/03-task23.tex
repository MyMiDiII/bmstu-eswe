\vspace{-0.5cm}
\section{Оптимизация по срокам реализации}

Критический путь показан на рисунке~\ref{img:lab03-task03-01}.
Было выявлено, что самыми длительными задачами на критическом пути являлись те,
на которые назначены программисты, при этом каждую из этих задач выполняли один
или два программиста из четырех, работающих над проектом, поэтому для
оптимизации критического пути было принято решение назначить всех четырех
программистов на каждую из задач, которую они могли выполнять.

\imgw{lab03-task03-01}{h!}{15.5cm}{Критический путь}

Результаты оптимизации представлены на рисунке~\ref{img:lab03-task03-03}.

\imgw{lab03-task03-03}{h!}{15.5cm}{Результаты оптимизации параметров проекта}

Таким образом, удалось достичь сокращения планируемого срока выполнения
до~19.07.2023, что вкладывается в установленные сроки реализации. При этом за
счет сокращения числа совещаний и времени выполнения задач, что дало сокращение
работы малостоящих ресурсов, удалось уменьшить затраты до 48~509~рублей.

\section{Анализ затрат и трудозатрат по структурным группам \mbox{ресурсов}}

Диагрммы затрат и трудозатрат до и после оптимизации приведены на
рисунках~\ref{img:lab02-task03}~и~\ref{img:lab03-task03} соответственно.

\imgw{lab02-task03}{h!}{15cm}{Соотношения затрат и трудозатрат до оптимизации}

\imgw{lab03-task03}{h!}{15cm}{Соотношения затрат и трудозатрат после оптимизации}

 По данным диаграммам видно, что на <<Программирование>> уменьшилось количество
 затрат и в то же время увеличилось количество трудозатрат, однако увеличились
 затраты на <<Аналитику>>, но при этом уменьшились трудозатраты на аренду
 сервера, что послужило причиной снижения затрат на нее.


