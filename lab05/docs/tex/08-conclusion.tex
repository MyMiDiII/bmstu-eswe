\section{Вывод}

По методу освоенного объема было определено, что проект выполняется с отставанием, однако укладывается в смету при этом отсутствует перерасход средств, что позволяет ликвидировать отставание.

С помощью отчетов было определено, что наибольшую потребность в деньгах руководитель проекта будет испытывать на 25~неделе года. С превышением затрат выполняются три задачи, и два ресурса, однако за счет изменений правил посещения совещаний наблюдается снижение затрат на них и ресурсы на них назначенные.

Также использование декомпозиции на базе каскадной модели жизненного цикла показало уменьшение времени и затрат на проект.