\clearpage
\section{Задание 1: Работа с таблицей освоенного объема}

\subsection{Прямые и косвенные затраты}

Для анализа прямых (фиксированных) и косвенных затрат воспользуемся таблицей затрат~(рисунок~\ref{img:lab05-01}).

\imgw{lab05-01}{h!}{17cm}{Таблица затрат на дату отчета}

На дату отчета общие фактически затраты составляют 14~945~рублей. Фактические прямые затраты не представлены в таблице, однако они начисляются пропорционально проценту выполнения задач. Проведя несложные вычисления получаем следующие фактические прямые затраты:

\begin{itemize}
    \item работа 2 --- $1000 \cdot 86\% = 860$~рублей;
    \item работа 8 --- $1000 \cdot 20\% = 200$~рублей;
    \item работа 12 --- $1000 \cdot 43\% = 430$~рублей;
\end{itemize}

Таким образом, фактические прямые затраты на весь проект составляют~1520~рублей~(10\%), откуда следует, что фактические косвенные затраты~(на использование ресурсов) составляют~13~425~рублей~(90\%).

\subsection{Освоенный объем}

Таблица освоенного объема, выведенная с помощью \texttt{Вид -> Таблицы -> Другие таблицы... -> Освоенный объем}, представлена на рисунке~\ref{img:lab05-02}.

\imgw{lab05-02}{h!}{17cm}{Таблица освоенного объема}

Таблица освоенного объема содержит следующие поля:

\begin{itemize}
    \item \textbf{запланированный объем (ЗО)} --- это те средства, которые были бы затрачены на выполнение задачи в период с начала проекта до выбранной даты отчета, если бы задача точно соответствовала графику и смете;
    \item \textbf{освоенный объем (ОО)} --- это те средства, которые были бы затрачены на выполнение задачи с самого начала проекта до выбранной даты отчета, если бы фактически выполненная работа оплачивалась согласно смете, т.~е.~это фактическое количество рабочих часов, оплачиваемых по сметным ставкам;
    \item \textbf{фактические затраты (ФЗ)} --- это средства, фактически потраченные на выполнение задачи в период с начала проекта до выбранной даты отчета, т.е. это фактическая стоимость задачи или фактическая ставка, умноженная на фактические часы;
    \item \textbf{отклонение от календарного плана (ОКП)} --- сравнивает сметную стоимость плановой и выполненной работы и позволяет вычислить несоответствие сметы, вызванное исключительно различиями между плановым и фактическим объемом работы;
    \item \textbf{отклонение по стоимости (ОПС)} --- сравнивает сметную и фактическую стоимость выполненной работы и позволяет выделить несоответствие сметы, вызванные разницей стоимости ресурсов;
    \item \textbf{предварительная оценка по завершении (ПОПЗ)} ---  отображает ожидаемые общие затраты для задачи, расчет которых основан на предположении, что оставшаяся часть работы будет выполнена в точном соответствии со сметой;
    \item \textbf{затраты по базовому плану (БПЗ)} --- отражает фиксированные затраты и стоимость ресурсов согласно базовому плану;
    \item \textbf{отклонение по завершению (ОПЗ)} --- разность между БПЗ и ПОПЗ.
\end{itemize}

Проанализируем таблицу освоенного объема нашего проекта по описанным полям:

\begin{itemize}
    \item $\text{ЗО} = 17~015.46$ рублей;
    \item $\text{ОО} = 14~562.16$ рублей;
    \item $\text{ФЗ} = 12~216.28$ рублей;
    \item $\text{ОКП} = \text{ОО} - \text{ЗО} = 14~562.16 - 17~015.46 = -2~453.3~\text{ рублей} < 0$ --- проект выполняется с отставанием;
    \item $\text{ОПС} = \text{ОО} - \text{ФЗ} = 14~562.16 - 12~216.28 = 2~345.88 \text{ рублей} > 0$ --- проект укладывается в смету, за счет этих средств можно выделить дополнительные ресурсы, чтобы ликвидировать отставание;
    \item $\text{ПОПЗ} = 40~695.04$ рублей;
    \item $\text{БПЗ} = 48~509.69$ рублей;
    \item $\text{ОПЗ} = \text{БПЗ} - \text{ПОПЗ} = 48~509.69 - 40~695.04 = 7~814.65 \text{ рублей} > 0$ --- отсутствует перерасход средств.
\end{itemize}

Отставание проекта ($\text{ОКП} < 0$) обусловлено завершением двух задач на неделю и 10~дней позже. Положительное отклонение по стоимости ($\text{ОПС} > 0$) произошло за счет исключения некоторых ресурсов с определенных совещаний в соответсвии с введенными с 10~апреля правилами посещения.