\section{Задание 2: Работа с отчетами проекта}

\subsection{Потребность в деньгах}

Получим отчет о затратах по неделям с помощью отчета \texttt{Отчет -> Наглядные отчеты -> Отчет о бюджетной стоимости}, представленного на рисунке~\ref{img:lab05-03}.

\imgw{lab05-03}{h!}{17cm}{Понедельный отчет о затратах}

Как видно из рисунка, наибольшую потребность в деньгах руководитель проекта будет испытывать на 25~неделе~(года), на которую приходятся задачи дорогостоящего <<Ведущего программиста>>, задачи четырех <<Программистов>>, а также <<3D-аниматора>> и <<Художника-дизайнера>>, то есть в эту неделю работают большее число ресурсов (в том числе и дорогостоящих) по сравнению с другими неделями. Такая же ситуация наблюдалась в базовом плане только на 23~неделе, но при вводе фактических параметров данные задачи сместились на 2~недели.

Также высокая потребность в деньгах у руководителя возникала на неделях~12-13, когда работал <<Системный аналитик>> --- самый дорогостоящий ресурс.

\subsection{Превышение затрат}

Превышение затрат по задачам и ресурсам в виде графиков и таблиц представлены на рисунках~\ref{img:lab05-04}-\ref{img:lab05-05} и \ref{img:lab05-06}-\ref{img:lab05-07} соответственно.

Наибольшее превышение затрат произошло на задаче \texttt{Построение базы объектов} за счет завершения работы~9 на 10~дней позже и увеличения времени работы сервера. Превышение же затрат на задачах \texttt{Создание ядра GIS} и \texttt{Тестирование сайта} возникло за счет увеличения ставки <<Ведущего программиста>> на 10\% после курсов повышения квалификации. Уменьшение затрат на совещания произошло за счет новых правил посещения совещаний с 10~апреля.

\imgw{lab05-04}{h!}{11cm}{Превышение затрат по задачам. График}

\imgw{lab05-05}{h!}{11cm}{Превышение затрат по задачам. Таблица}

Увеличение затрат на <<Ведущего программиста>> произошло за счет величения ставки <<Ведущего программиста>> на 10\% после курсов повышения квалификации, на <<Сервер>> --- за счет завершения работы~9 на 10~дней позже и увеличения времени его работы.

Уменьшение затрат на ресурсы с отрицательным отклонением по стоимости произошло за счет уменьшения посещаемых ими совещаний.

\imgw{lab05-06}{h!}{11cm}{Превышение затрат по ресурсам. График}

\imgw{lab05-07}{h!}{11cm}{Превышение затрат по ресурсам. Таблица}
