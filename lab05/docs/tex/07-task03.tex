\section{Задание 3: Анализ вариантов декомпозиции работ в проекте}

Декомпозиция с результатами выполнения лабораторной работы №2, представлена на рисунке~\ref{img:lab05-08}.

\imgw{lab05-08}{h!}{17cm}{Заданная декомпозиция без выравниваний и оптимизаций (результат ЛР2)}

В соответствии с каскадной моделью жизненный цикл программного обеспечения состоит из следующих этапов:

\begin{itemize}
    \item анализ;
    \item проектирование;
    \item разработка;
    \item тестирование;
    \item поддержка.
\end{itemize}

Новая декомпозиция на основе данной модели с новыми связями задач, переназначением фиксированных затрат и сервера на другие задачи представлена на рисунке~\ref{img:lab05-09}.

\imgw{lab05-09}{h!}{17cm}{Новая декомпозиция без выравниваний и оптимизаций}

Как видно из рисунков, при заданной декомпозиции без выравниваний и оптимизаций проект оканчивается по плану с запаздыванием 18.09.2023 при затратах в 48~126~рублей. При новой же декомпозиции при тех же условиях проект заканчивается с опережением 16.08.2023 при затратах 42~724~рубля.

Сокращение сроков вызвано отсутсвием ожидания завершения создания рабочей версии ядра для выполнения работ по наполнению сайта и базы данных. Уменьшение затрат связано с сокращением времени работы сервера за счет исключения его с задачи по анализу и построению структуры базы объектов.

Также проведем сравнение с результатами лабораторной работы~№3~(рисунок~\ref{img:lab05-10}). Для этого проведем все действия из нее с новой декомпозицией, включая оптимизацию затрат (по совещаниям) и критического пути (также путем назначения всех программистов на все задачи, связанные с программированием). Результат оптимизации параметров проекта с новой декомпозии приведен на рисунке~\ref{img:lab05-11}.

\imgw{lab05-10}{h!}{17cm}{Заданная декомпозиция после оптимизации (результат ЛР3)}

\imgw{lab05-11}{h!}{17cm}{Новая декомпозиция после оптимизации}

\clearpage

Как видно из рисунков, при заданной декомпозиции после выравнивания и оптимизации проект оканчивается по плану 19.07.2023 при затратах в 48~509~рублей. При новой же декомпозиции при тех же условиях проект заканчивается раньше 07.07.2023 при затратах 45~156~рубля.

Таким образом, введение новой декомпозиции позволило найти новые связи между задачами и определить работы, для которых были выделены не требующиеся для них ресурсы, что в свою очередь позволило сократить длительность и затраты проекта.
