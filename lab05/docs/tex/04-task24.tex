\clearpage
\section{Актуализация параметров проекта}

\subsection{Текущие параметры проекта}

Дата отчета: 17.04.2023

На эту дату завершились все работы, которые должны были завершиться, кроме:

\begin{itemize}
    \item работа 6 завершилась на неделю позже;
    \item работа 9 завершилась на 10 дней позже;
    \item работа 17 выполнена на 10\%;
    \item с 1~апреля на 2~недели ведущий программист уехал на курсы повышения
	  квалификации, после возвращения с которых его заработная плата
	  поднята на~10\%;
    \item с 10~апреля совещания посещают только те сотрудники, которые
        выполняют работы на данный момент или будут выполнять работу через
        2~недели и меньше; после окончания своих работ сотрудники совещания не
        посещают.
\end{itemize}

\subsection{Результаты анализа хода выполнения проекта}

Диаграмма Ганта проекта с учетом фактических параметров и линией прогресса представлена на рисунке~\ref{img:lab04-09}. Отклонение от базового плана представлено на рисунке~\ref{img:lab04-10}.

\imgw{lab04-09}{h!}{17cm}{Диаграмма Ганта проекта с учетом фактических параметров}

\imgw{lab04-10}{h!}{17cm}{Отклонения от базового плана}

При учете фактических параметров отклонение от базового плана составило 9~дней в сторону увеличения длительности проекта и 190 рублей затрат также в сторону увеличения, однако полученные данные времени и затрат допустимы с точки зрения первоначальных требований проекта: дата окончания --- 28.07.2023 (месяц запаса), затраты --- 48~700~рублей (1~300~рублей запаса бюджета).