\section{Выполнение тренировочного задания}

Задание по варианту №2:

\begin{enumerate}
    \item Дополнить временной план проекта, подготовленный на предыдущем этапе
    (лабораторная работа № 1), информацией о ресурсах и определить стоимость
    проекта.

    \item Для этого заполнить ресурсный лист в программе MS Project, принимая во
        внимание, что к реализации проекта привлекается не более
        12~исполнителей. 

    \item Предусмотреть, что стандартная ставка ресурса составляет
        250~руб./день.

    \item Произвести назначение ресурсов на задачи в соответствии с таблицей. С
        учетом того, что квалификация ресурсов одинаковая, при назначении
        ресурсов использовать процент загрузки. 

    {
    \captionsetup{format=hang,justification=raggedleft,
                  singlelinecheck=off,width=17cm}
    \begin{longtable}[Hc]{|c|c|}
    \caption{Назначение ресурсов}\\
        \hline
        Название работы & Количество исполнителей (чел.)\\
        \hline
        Работа A & 1\\
        \hline
        Работа B & 4\\
        \hline
        Работа C & 5\\
        \hline
        Работа D & 2\\
        \hline
        Работа E & 3\\
        \hline
        Работа F & 3\\
        \hline
        Работа G & 5\\
        \hline
        Работа H & 5\\
        \hline
        Работа I & 2\\
        \hline
        Работа J & 6\\
        \hline
    \end{longtable}
    }

    \item К получившемуся после назначения ресурсов бюджету проекта добавить
        5~тыс.~рублей фиксированных затрат. Кроме того, с понедельника второй
        недели реализации проекта выделить средства на приобретение расходных
        материалов для выполнения работ А, В и С из расчета 1~тыс.~рублей в
        неделю.
\end{enumerate}

Заполненный ресурсный лист, являющийся результатом выполнения пунктов 2-3,
приведен на рисунке~\ref{img:lab02-test01}.

\imgw{lab02-test01}{h!}{17cm}{Ресурсный лист проекта тренировочного задания}

Назначение ресурсов на задачи производится с помощью одноименного окна каждой
задачи. Пример назначения приведен на рисунке~\ref{img:lab02-test02}.

\imgw{lab02-test02}{h!}{9cm}{Назначение ресурсов на задачу <<Работа J>>}

Результат назначения ресурсов на все задачи представлен на
рисунке~\ref{img:lab02-test03}.

\imgw{lab02-test03}{h!}{14.5cm}{Диаграмма Ганта проекта тренировочного задания с
назначенными трудовыми ресурсами}

Добавление фиксированных затрат к бюджету проекта произведено через вид
\texttt{Вид -> Таблица -> Затраты} с помощью установки значения колонки
\texttt{Фиксированные затраты} суммарной задачи проекта, результат чего
приведен на рисунке~\ref{img:lab02-test04}.

\imgw{lab02-test04}{h!}{17cm}{Установка фиксированных затрат}

Для выделения средств на приобретение расходных материалов для выполнения работ
А, В и С из расчета 1 тыс. рублей в неделю с понедельника второй недели
реализации проекта происходит добавление еще одного трудового (так как
стоимость исчиляется на основе времени) ресурса на каждую задачу (максимальные
единицы 300\%) с указанием доступности ресурса с 06.03.2023 и стандартной
ставкой 1000 руб./нед.~(рисунок~\ref{img:lab02-test05}).

\imgw{lab02-test05}{h!}{17cm}{Итоговый ресурсный лист проекта тренировочного
задания}

Аналогично назначению исполнителей происходит назначение ресурса <<расходные
материалы>> на работы A, B и
C~(рисунки~\ref{img:lab02-test06}-\ref{img:lab02-test07}).

\imgw{lab02-test06}{h!}{10cm}{Назначение ресурса <<расходные материалы>> на
<<Работу A>>}

\imgw{lab02-test07}{h!}{17cm}{Диаграмма Ганта с назначенным ресурсом
<<расходные материалы>>}

\section*{Вывод}

В представлении \texttt{Диаграмма Ганта} у задач, содержащий ресурсы с
превышением доступности, то есть у тех, которые одновременно используют больше
ресурсов, чем доступно, в столбце \texttt{Индикаторы} высвечивается
соответствующее предупреждение~(рисунок~\ref{img:lab02-test07}). В представлении
\texttt{Лист ресурсов} ресурсы, доступность которых превышена, отмечаются
красным цветом~(рисунок~\ref{img:lab02-test05}). Наглядно же посмотреть из-за
чего происходит нехватка ресурсов можно с помощью представления
\texttt{Визуальный оптимизатор ресурсов}~(рисунок~\ref{img:lab02-test08}). 

\imgw{lab02-test08}{h!}{17cm}{Визуальное представление наложения ресурсов}

В случае рассматриваемого проекта по рисунку~\ref{img:lab02-test08} наблюдается
нехватка исполнителей при одновременном выполнении работ C, D, J и B, G, H.
Также для работы C расходные материалы поступают на три дня позже начала
выполнения задачи. При этом планируемый бюджет проекта \texttt{119 750}~рублей.
