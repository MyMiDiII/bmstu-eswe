\section{Выполнение тренировочного задания}

Задание по варианту №2.

Дополнить временной план проекта информацией о ресурсах учитывая, что к
реализации проекта привлекается не более 12~исполнителей, а стандартная ставка
ресурса составляет 250~руб./день. Назначение ресурсов на задачи происходит в
соответствии с таблицей~\ref{tab:01}; квалификация ресурсов одинаковая.

{
\fontsize{12pt}{12pt}\selectfont
\captionsetup{format=hang,justification=raggedleft,
              singlelinecheck=off,width=17cm}
\begin{longtable}[Hc]{|c|c|}
\caption{\label{tab:01}Назначение ресурсов}\\
    \hline
    Название работы & Количество исполнителей (чел.)\\
    \hline
    Работа A & 1\\
    \hline
    Работа B & 4\\
    \hline
    Работа C & 5\\
    \hline
    Работа D & 2\\
    \hline
    Работа E & 3\\
    \hline
    Работа F & 3\\
    \hline
    Работа G & 5\\
    \hline
    Работа H & 5\\
    \hline
    Работа I & 2\\
    \hline
    Работа J & 6\\
    \hline
\end{longtable}
}

В бюджете проекта необходимо учесть 5~тыс.~руб.  фиксированных затрат. Кроме
того, с понедельника второй недели реализации проекта выделяются средства на
приобретение расходных материалов для выполнения работ А,~В~и~С из расчета
1~тыс.~руб. неделю.  Определить стоимость проекта.

Заполненный ресурсный лист приведен на рисунке~\ref{img:lab02-test01}.

\imgw{lab02-test01}{h!}{17cm}{Ресурсный лист проекта тренировочного задания}

Назначение ресурсов на задачи производится с помощью одноименного окна каждой
задачи. Пример назначения приведен на рисунке~\ref{img:lab02-test02}.

\imgw{lab02-test02}{h!}{10cm}{Назначение ресурсов на задачу <<Работа J>>}

Результат назначения ресурсов на все задачи представлен на
рисунке~\ref{img:lab02-test03}.

\imgw{lab02-test03}{H}{16cm}{Диаграмма Ганта проекта тренировочного задания с
назначенными трудовыми ресурсами}

Добавление фиксированных затрат к бюджету проекта произведено через вид
\texttt{Вид -> Таблица -> Затраты} с помощью установки значения колонки
\texttt{Фиксированные затраты} суммарной задачи проекта, результат чего
приведен на рисунке~\ref{img:lab02-test04}.

\imgw{lab02-test04}{h!}{17cm}{Установка фиксированных затрат}

Для выделения средств на приобретение расходных материалов для выполнения работ
А, В и С из расчета 1 тыс. рублей в неделю с понедельника второй недели
реализации проекта происходит добавление еще одного трудового (так как
стоимость исчисляется на основе времени) ресурса на каждую задачу (максимальные
единицы 300\%) с указанием доступности ресурса с 06.03.2023 и стандартной
ставкой 1000 руб./нед.~(рисунок~\ref{img:lab02-test05}).

\imgw{lab02-test05}{h!}{17cm}{Итоговый ресурсный лист проекта тренировочного
задания}

Аналогично назначению исполнителей происходит назначение ресурса <<расходные
материалы>> на работы A, B и
C~(рисунки~\ref{img:lab02-test06}-\ref{img:lab02-test07}).

\imgw{lab02-test06}{h!}{10cm}{Назначение ресурса <<расходные материалы>> на
<<Работу A>>}

\imgw{lab02-test07}{h!}{15cm}{Диаграмма Ганта с назначенным ресурсом
<<расходные материалы>>}

\vspace{-0.5cm}
\section*{Вывод}

В представлении \texttt{Диаграмма Ганта} у задач, содержащий ресурсы с
превышением доступности, то есть у тех, которые одновременно используют больше
ресурсов, чем доступно, в столбце \texttt{Индикаторы} высвечивается
соответствующее предупреждение~(рисунок~\ref{img:lab02-test07}). В представлении
\texttt{Лист ресурсов} ресурсы, доступность которых превышена, отмечаются
красным цветом~(рисунок~\ref{img:lab02-test05}). Наглядно же посмотреть из-за
чего происходит нехватка ресурсов можно с помощью представления
\texttt{Визуальный оптимизатор ресурсов}~(рисунок~\ref{img:lab02-test08}). 

\imgw{lab02-test08}{h!}{17cm}{Визуальное представление наложения ресурсов}

В случае рассматриваемого проекта по рисунку~\ref{img:lab02-test08} наблюдается
нехватка исполнителей при одновременном выполнении работ C, D, J и B, G, H.
Также для работы C расходные материалы поступают на три дня позже начала
выполнения задачи. При этом планируемый бюджет проекта 119~750~рублей.
