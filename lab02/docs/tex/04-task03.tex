\section{Задание 3: Анализ затрат по группам ресурсов}

Структуризации затрат и трудозатрат по группам ресурсов приведены на
рисунках~\ref{img:lab02-task03-01},~\ref{img:lab02-task03-03} соответственно.

\imgw{lab02-task03-01}{h!}{14.5cm}{Структуризация затрат по группам ресурсов}

\imgw{lab02-task03-03}{h!}{14.5cm}{Структуризация трудозатрат по группам ресурсов}

Круговые диаграммы о затратах и трудозатратах по структурным группам ресурсов
представлены на рисунках~\ref{img:lab02-task03-02},~\ref{img:lab02-task03-04}
соответственно.

\imgw{lab02-task03-02}{h!}{10cm}{Распределение затрат по структурным группам
ресурсов}

\imgw{lab02-task03-04}{h!}{10cm}{Распределение трудозатрат по структурным группам
ресурсов}

Таким образом, при 29\% трудозатрат затраты по группе <<Программирование>> 50\%,
в то время как по группе <<Ввод данных>> с трудозатратами в
25\% затраты в 4.5~раза ниже и составляют 11\%. Также при примерно равных
затратах на <<Аналитику>>, <<Аренду>> и <<Ввод данных>> трудозатраты на
<<Аренду>> в среднем в 14~раз меньше, чем на <<Аренду>> и <<Ввод данных>>. По
оставшимся группам трудозатраты соразмерным затратам.

По приведенным соотношения затрат и трудозатрат по определенным группам можно
судить о ценности труда для проекта определенных сотрудников или работы
материального ресурса, что дает возможность при несовпадении приносимого в
проект вклада и затрат переоценить и изменить стоимость тех или иных ресурсов.
